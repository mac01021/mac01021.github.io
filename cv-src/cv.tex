%%%%%%%%%%%%%%%%%%%%%%%%%%%%%%%%%%%%%%%%%
% "ModernCV" CV and Cover Letter
% LaTeX Template
% Version 1.1 (9/12/12)
%
% This template has been downloaded from:
% http://www.LaTeXTemplates.com
%
% Original author:
% Xavier Danaux (xdanaux@gmail.com)
%
% License:
% CC BY-NC-SA 3.0 (http://creativecommons.org/licenses/by-nc-sa/3.0/)
%
% Important note:
% This template requires the moderncv.cls and .sty files to be in the same 
% directory as this .tex file. These files provide the resume style and themes 
% used for structuring the document.
%
%%%%%%%%%%%%%%%%%%%%%%%%%%%%%%%%%%%%%%%%%

%----------------------------------------------------------------------------------------
%	PACKAGES AND OTHER DOCUMENT CONFIGURATIONS
%----------------------------------------------------------------------------------------

\documentclass[11pt,a4paper,sans]{moderncv} % Font sizes: 10, 11, or 12; paper sizes: a4paper, letterpaper, a5paper, legalpaper, executivepaper or landscape; font families: sans or roman

\moderncvstyle{classic} % CV theme - options include: 'casual' (default), 'classic', 'oldstyle' and 'banking'
\moderncvcolor{grey} % CV color - options include: 'blue' (default), 'orange', 'green', 'red', 'purple', 'grey' and 'black'
\usepackage[T1]{fontenc}
\usepackage{lmodern}
\usepackage{lipsum} % Used for inserting dummy 'Lorem ipsum' text into the template

\usepackage[scale=0.85]{geometry} % Reduce document margins
\setlength{\hintscolumnwidth}{4cm} % Uncomment to change the width of the dates column
\setlength{\makecvtitlenamewidth}{10cm} % For the 'classic' style, uncomment to adjust the width of the space allocated to your name

%----------------------------------------------------------------------------------------
%	NAME AND CONTACT INFORMATION SECTION
%----------------------------------------------------------------------------------------

\firstname{Matthew} % Your first name
\familyname{Coolbeth} % Your last name

% All information in this block is optional, comment out any lines you don't need
\title{Software Engineer and  Computer Scientist}
\address{2746 Hebron Avenue}{Glastonbury, CT 06033}

\phone{(860) 420 2024}
\email{coolbeth@gmail.com}

% The first argument is the url for the clickable link, the second argument is the 
% url displayed in the template - this allows special characters to be displayed such
% as the tilde in this example
%\homepageone{staff.org.edu/~jsmith}{staff.org.edu/$\sim$jsmith} 
%\homepageone{https://www.linkedin.com/in/mac01021}
%			{https://www.linkedin.com/in/mac01021} 
%\homepagetwo{https://github.com/mac01021}
%			{https://github.com/mac01021}
%%\homepagethree{https://play.google.com/store/apps/developer?id=mac01021}	 
%%			  {https://play.google.com/store/apps/developer?id=mac01021} 


%\homepagefour{staff.org.edu/~jsmith}{staff.org.edu/$\sim$jsmith} 

%\extrainfo{More stuff}
%\photo[70pt][0.4pt]{pictures/picture} % The first bracket is the picture height, the second is the thickness of the frame around the picture (0pt for no frame)
%\quote{"A witty and playful quotation" - John Smith}

%----------------------------------------------------------------------------------------

\begin{document}





%----------------------------------------------------------------------------------------
%	COVER LETTER
%----------------------------------------------------------------------------------------

%% To remove the cover letter, comment out this entire block
%
%\clearpage
%
%%\recipient{HR Departmnet}{Corporation\\123 Pleasant Lane\\12345 City, State} % Letter recipient
%\recipient{~}{~}
%\date{\today} % Letter date
%\opening{To whom this may concern,} % Opening greeting
%\closing{Sincerely,} % Closing phrase
%\enclosure[Attached]{r\'{e}sum\'{e}{}} % List of enclosed documents
%
%\makelettertitle % Print letter title
%
%%%\lipsum[1-3] % Dummy text
%This letter is written with interest in the position of Software Engineer III, Data and Platform Technology
%advertised on the ESPN Careers website.
%
%I am a versatile software engineer with a graduate degree in computer science and a broad background in application
%development. My experience includes a variety of projects that range from prototyping forecasting systems based on 
%machine learning and signal processing to the development and maintenance of large, custom, enterprise web applications.
%I have a strong grasp of modern web and database technologies, and am well versed in a variety of programming languages
%and environments.
%
%After working for the last four years at the University of Connecticut, building
%enterprise workflow applications, I am eager to move to a position that offers
%increased technical challenge and responsibility, and allows me to work with larger teams on
%systems that solve harder problems. 
%I am excited about finding an opportunity to work on ESPN's data processing
%infrastructure and would very much like to learn more about the details of this
%position.  
%
%Attached to this letter, you will find a r\'{e}sum\'{e} detailing my background
%and qualifications.  Please feel free to contact me at any time if you have
%questions about its content, or would like to meet me to discuss in greater depth.
%I am, as I have said, eager to learn more about the details of this position,
%and I hope that I will hear from you soon.
%
%Thank you for your time.
%
%
%
%\makeletterclosing % Print letter signature
%
%
%
%\newpage
%


%----------------------------------------------------------------------------------------
%	CV
%--------------------------------------------------------------------------------------
\makecvtitle % Print the CV title


\vspace{-1.5cm}
\section{Professional Experience}

\cventry{Feb 2015 -- Present}{Software Engineer}{\textsc{ESPN}}{Bristol, CT}{}{
	Building a data platform, using Apache Kafka and friends for streams
	processing.  We have constructed a high-availability distributed
	system, using Zookeeper for service coordination, with microservices
	communicating via Apache-Avro-formatted messages using Kafka as a
	message bus.  We are a JVM shop, with most software written in the 
	Java programming language.  Heavy use of Git and Maven.
}

\cventry{Aug 2010 -- Feb 2015}{Applications Developer}{\textsc{University of Connecticut}}{Storrs, CT}{}{
	Served on an agile software development team building custom enterprise software for university
	staff.  Followed a SCRUM-style methodology with two-week sprints.
	Used a branch-based (SVN) development process incorporating continual code review and
	extensive automated testing.
	~\\
	The majority of our products were web applications, built using Python MVC frameworks, developed and hosted on
	Linux, and using MS SQL Server as a backing data store.
}


\cventry{Jan 2010 -- Aug 2010}{Software Developer}{\textsc{QueBIT Consulting, LLC}}{Wilton, CT}{}{
	\begin{itemize}
		\item Helped to prototype a web application for double-entry book keeping based on the
			IBM COGNOS TM1 database system.
		\item Worked on Excel plugin to populate a journal template in an excel spreadsheet from TM1.
		\item These products later evolved into QueBIT's "ControlWORQ" product.
		\item Used C\#, ASP.NET, SOAP, MySQL, TM1, VBA, APIs for MS Excel.
	\end{itemize}
}

\cventry{Sep 2009 -- Dec 2009}{Software Developer (Contractor) }{\textsc{Pfizer, Inc}}{Groton, CT}{}{
	Developed a Java EE web application to process sets of nucleotide sequences for research biologists.
	\begin{itemize}
		\item Collaborated with genomics researchers to develop and refine application requirements.
		\item Prototyped application to reflect changing requirements in an agile manner.
		\item The application accepts zip files produced by DNA sequencers, converts each nucleotide
			sequence to an amino acid sequence, executes an external process to generate a
			multi-sequence alignment, and then generates a report depicting each alignment.
		\item Used Java 6 EE, Tomcat, JSP, BioJava, HMMER, JQuery.
	\end{itemize}
}

\cventry{Jan 2008 -- Aug 2009}{Research Specialist}{\textsc{University of Connecticut}}{Storrs, CT}{}{
	Worked in an interdepartmental research group to build two energy-demand-forecasting
			systems for ISO New England.
	\begin{itemize}
		\item Prototyped forecasting methods incorporating signal processing and
			machine learning techniques (wavelet decomposition \& neural networks).
		\item Primary Java developer for two forecasting engines: one short-term (days) and one
			very short term (hourse).  The later system was later open-sourced.
		\item Used Java 6, Swing, C++, MATLAB, Oracle, Perst (embedded database), Git, CVS.
	\end{itemize}
}

\cventry{Aug 2007 -- Dec 2007}{Teaching Assistant}{\textsc{University of Connecticut}}{Storrs, CT}{}{
	\begin{itemize}
		\item Instructed weekly lab activity for a sophomore-level course on object oriented design.
		\item Tutored undergraduate students in areas related to computer science and mathematics.
	\end{itemize}
}



\section{Education}

\cventry{2009}{MS - Computer Science \& Engineering}{University of Connecticut}{}{}{}  % Arguments not required can be left empty
\cventry{2007}{BS - Computer Science}{University of Connecticut}{}{}{}



\newpage
\section{Programming Tools \& Practices}

\cvitem{Version Control:} {\textsc{Git, Mercurial, Subversion, CVS}}
\cvitem{Programming Languages:}{\textsc{C/C++, Clojure, Go, Java, Javascript, Python, Scala, SQL } et cetera}
\cvitem{Database Systems:} {\textsc{MS SQL Server, MySQL, Oracle, SQLite, Cassandra}}
\cvitem{Web Technologies:}{\textsc{AJAX, REST, Websockets}}
\cvitem{Other Technologies:} {\textsc{Kafka, Zookeeper, Spark}}
\cvitem{Engineering Practices:}{\textsc{Agile, SCRUM, Continuous Integration, Test-Driven Development}}

{\em I am probably omitting something of interest.  Ask me about your favorite tool or practice.}

\begin{thebibliography}{1}

	\bibitem{stlf} {\em Short-term Load Forecasting: Similar Day-Based Wavelet Neural Networks}
		(Y. Chen, P. B. Luh, Y. Zhao, L. D. Michel, M. A. Coolbeth, P. B. Friedland, \& S. J. Rourke)
		{\em In IEEE Transactions on Power Systems, volume 25, 2010}.

	\bibitem{vstlf} {\em Very Short-term Load Forecasting: Multilevel Wavelet Neural Networks with Data
		Pre-filtering}
		(C. Guan, P. B. Luh, L. D. Michel, M. A. Coolbeth, Y. Zhao, Y. Chen, C. J. Manville, P. B. Friedland,
		\& S. J. Rourke)
		{\em In Proceedings of the 2009 PESGM, 2009}.

	\bibitem{data-collection} {\em Data Collection with Multiple Sinks in Wireless Sensor Networks} 
		(S. Chen, M. Coolbeth, H. Dinh, Y. A. Kim, \& B. Wang)
		{\em In Proceedings of the 2009 conference on Wireless Algorithms, Systems, and Applications, 2009}.

\end{thebibliography}


\end{document}

